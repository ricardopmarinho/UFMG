\documentclass[11pt,a4paper]{book}
\usepackage[brazilian]{babel}
\usepackage[utf8]{inputenc}
\usepackage[T1]{fontenc}
\usepackage[inline]{enumitem}
\usepackage{xcolor}
\usepackage{listings}
\usepackage{graphicx}
\usepackage{multicol}
\usepackage{amsmath}
\usepackage{amssymb}

\definecolor{mGreen}{rgb}{0,0.6,0}
\definecolor{mGray}{rgb}{0.5,0.5,0.5}
\definecolor{mPurple}{rgb}{0.58,0,0.82}
\definecolor{backgroundColour}{rgb}{0.95,0.95,0.92}

\lstdefinestyle{CStyle}{
    backgroundcolor=\color{backgroundColour},   
    commentstyle=\color{mGreen},
    keywordstyle=\textbf{\color{black}},
    numberstyle=\tiny\color{mGray},
    stringstyle=\color{mPurple},
    basicstyle=\footnotesize,
    breakatwhitespace=false,         
    breaklines=true,                 
    captionpos=b,                    
    keepspaces=true,                 
    numbers=left,                    
    numbersep=5pt,                  
    showspaces=false,                
    showstringspaces=false,
    showtabs=false,                  
    tabsize=2,
    frame=single,
    escapeinside={(*}{*)},
    language=C
}

\makeatletter
% This command ignores the optional argument for itemize and enumerate lists
\newcommand{\inlineitem}[1][]{%
\ifnum\enit@type=\tw@
    {\descriptionlabel{#1}}
  \hspace{\labelsep}
\else
  \ifnum\enit@type=\z@
       \refstepcounter{\@listctr}\fi
    \quad\@itemlabel\hspace{\labelsep}
\fi}
\makeatother

\newcommand{\onestaritem}{\refstepcounter{enumi}\item[$*$\theenumi.]}
\newcommand{\twostaritem}{\refstepcounter{enumi}\item[$**$\theenumi.]}

\title{Lista 8: Fundamentos Estatísticos para Ciência dos Dados}
\author{Ricardo Pagoto Marinho}

\begin{document}
\maketitle
	\begin{itemize}
		\item
		Para esta lista, um algoritmo semelhante ao apresentado na lista foi utilizado.
		Ele é apresentado a seguir:
		\begin{lstlisting}
library(imager)
im<-read.csv("zip.train",header = FALSE,sep="")
leio o arquivo de teste
mat_dig<-list()
for(i in 1:10){
    mat_dig[[i]]<-list()
}
crio uma matriz para armazenar as informacoes
sobre as imagens dos 10 digitos 
for(i in 1:nrow(im)){
    dig<-im[i,1]+1
    descubro o digito da linha
    pixels<-im[i,2:ncol(im)]
    pego os pixels da imagem do digito
    ind<-length(mat_dig[[dig]])+1
    descubro quantas imagens ja estao armazenadas para aquele
    digito
    mat_dig[[dig]][[ind]]<-matrix(pixels,ncol = 16,nrow = 16)
    armazeno os pixels da imagem daquele digito em uma matriz 16x16
}
como os digitos nao aparecem na mesma quantidade nas imagens
preciso saber qual a quantidade de vezes que o digito que
aparece mais vezes aparece
maior<-0
for(i in 1:length(mat_dig[])){
    if(length(mat_dig[[i]])>maior){
        maior<-length(mat_dig[[i]])
    }
}
mat_pixels<-matrix(0,nrow = 16*16,ncol=10*maior)
crio a matriz de pixels com o tamanho da imagem (16x16)
e a quantidade de imagens (10*maior)
for(i in 1:10){
    for(j in 1:length(mat_dig[[i]])){
        mat_pixels[,j+(i-1)*length(mat_dig[[i]])] = 
        stack(as.data.frame(mat_dig[[i]][[j]]))[,1]
    }
}
set.seed(123)
ind=sample(10,1:maior,replace = T)
indcol=ind+((1:10)-1)*maior
mat_test=mat_pixels[,indcol]
acho a imagem de teste
mat_pixels=mat_pixels[,-indcol]
mat_media=apply(mat_pixels, 1, mean)
mat_centrada=mat_pixels-mat_media
pca_pixels=prcomp(t(mat_centrada))
autovalores=(pca_pixels$sdev)^2
autovetores = pca_pixels$rot[ , 1:20]
exemplo utilizando 20 autovetores
auto_face=list()
for(i in 1:20){
    auto_face[[i]]=as.cimg(pca_pixels$rot[,i],x=16,y=16)
}
coef = t(autovetores) %*% (mat_test - mat_media)
coef_treino = t(autovetores) %*% (mat_pixels - mat_media)
coefmedio = matrix(0, ncol=10, nrow=20)
for(i in 1:10){
    coefmedio[,i] = apply(coef_treino[ ,(1+(i-1)*10) : (i*10)], 1, mean)
}
indproximo = numeric()
for(j in 1:10){
    indproximo[j] = which.min( apply((coefmedio - coef[,j])^2, 2, mean) )
}
		\end{lstlisting}
		\item
		\begin{tabular}{|c|c|c|c|c|c|c|c|c|c|}
		\hline
		0& 1& 2& 3&4 &5 &6 &7 &8 &9\\
		\hline
		0& 3&4 &2 &5 &6 &8 &1 &7 &0 \\
		\hline
		1&1&3 &5 &9 &1 &7 &2 &4 &4\\
		\hline
		0& 3&4 &2 &5 &6 &8 &1 &7 &0 \\
		\hline
		1&1&3 &5 &9 &1 &7 &2 &4 &4\\
		\hline
		4&0&2 &4 &9 &5 &4 &6 &8 &6 \\
		\hline
		0& 3&4 &2 &5 &6 &8 &1 &7 &0 \\
		\hline
		6& 0&2 &4 &9 &5 &4 &6 &8 &6 \\
		\hline
		1&1&3 &5 &9 &1 &7 &2 &4 &4\\
		\hline
		0& 3&4 &2 &5 &6 &8 &1 &7 &0 \\
		\hline		
		1&1&3 &5 &9 &1 &7 &2 &4 &4\\
		\hline
		\end{tabular}
	\end{itemize}
\end{document}