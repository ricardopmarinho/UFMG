\documentclass[11pt,a4paper]{book}
\usepackage[brazilian]{babel}
\usepackage[utf8]{inputenc}
\usepackage[T1]{fontenc}
\usepackage[inline]{enumitem}
\usepackage{xcolor}
\usepackage{listings}
\usepackage{graphicx}
\usepackage{multicol}
\usepackage{amsmath}
\usepackage{amssymb}

\definecolor{mGreen}{rgb}{0,0.6,0}
\definecolor{mGray}{rgb}{0.5,0.5,0.5}
\definecolor{mPurple}{rgb}{0.58,0,0.82}
\definecolor{backgroundColour}{rgb}{0.95,0.95,0.92}

\lstdefinestyle{CStyle}{
    backgroundcolor=\color{backgroundColour},   
    commentstyle=\color{mGreen},
    keywordstyle=\textbf{\color{black}},
    numberstyle=\tiny\color{mGray},
    stringstyle=\color{mPurple},
    basicstyle=\footnotesize,
    breakatwhitespace=false,         
    breaklines=true,                 
    captionpos=b,                    
    keepspaces=true,                 
    numbers=left,                    
    numbersep=5pt,                  
    showspaces=false,                
    showstringspaces=false,
    showtabs=false,                  
    tabsize=2,
    frame=single,
    escapeinside={(*}{*)},
    language=C
}

\makeatletter
% This command ignores the optional argument for itemize and enumerate lists
\newcommand{\inlineitem}[1][]{%
\ifnum\enit@type=\tw@
    {\descriptionlabel{#1}}
  \hspace{\labelsep}
\else
  \ifnum\enit@type=\z@
       \refstepcounter{\@listctr}\fi
    \quad\@itemlabel\hspace{\labelsep}
\fi}
\makeatother

\newcommand{\onestaritem}{\refstepcounter{enumi}\item[$*$\theenumi.]}
\newcommand{\twostaritem}{\refstepcounter{enumi}\item[$**$\theenumi.]}
\DeclareMathOperator*{\maxi}{max}

\title{Lista 11: Fundamentos Estatísticos para Ciência dos Dados}
\author{Ricardo Pagoto Marinho}

\begin{document}
\maketitle
	\begin{enumerate}
		\item
		\begin{itemize}
		
			\item
			
			Para o caso de $c(1|2)=c(2|1)$ e $\pi_1=\pi_2$, a comparação fica reduzida apenas à quantidade de indivíduos nas duas amostras, já que o que vai definir a região é qual função de densidade é a maior.
			
			\item 
			
			Para $\pi_1=0.001$, consequentemente $\pi_2=0.99$ e $c(1|2)=c(2|1)$, a função de classificação fica:
			
			\begin{eqnarray*}
				\frac{f_1(x)}{f_2(x)}>&\frac{\pi_2}{\pi_1}\\
				\frac{f_1(x)}{f_2(x)}>&\frac{0.99}{0.01}\\
				\frac{f_1(x)}{f_2(x)}>&99\\
				f_1(x)>&99f_2(x)
			\end{eqnarray*}
			
			Ou seja, $f_1(x)$ deve ser mais do que 99 vezes maior do que $f_2(x)$
			
			\item
			
			Neste caso, com $c(1|2)=\frac{c(2|1)}{10}$ e $\pi_1=0.001$ e $\pi_2=0.99$ temos: 
			
			\begin{eqnarray*}
				\frac{f_1(x)}{f_2(x)}>&\frac{c(1|2)}{c(2|1)}\frac{\pi_2}{\pi_1}\\
				\frac{f_1(x)}{f_2(x)}>&\frac{c(2|1)}{10}\frac{1}{c(2|1)}\frac{0.99}{0.01}\\
				\frac{f_1(x)}{f_2(x)}>&\frac{0.99}{0.1}\\
				\frac{f_1(x)}{f_2(x)}>&9.9\\
				f_1(x)>&9.9f_2(x)
			\end{eqnarray*}
			
			Ou seja, a regra do item anterior fica 10 vezes menor, já que aqui, $f_1(x)$ deve ser mais do que 9.9 vezes maior do que $f_2(x)$.
		\end{itemize}
		
		\item
		
		
	\end{enumerate}
\end{document}