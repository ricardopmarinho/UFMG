\documentclass[11pt,a4paper]{book}
\usepackage[brazilian]{babel}
\usepackage[utf8]{inputenc}
\usepackage[T1]{fontenc}
\usepackage[inline]{enumitem}
\usepackage{xcolor}
\usepackage{listings}
\usepackage{graphicx}
\usepackage{multicol}
\usepackage{amsmath}
\usepackage{amssymb}

\definecolor{mGreen}{rgb}{0,0.6,0}
\definecolor{mGray}{rgb}{0.5,0.5,0.5}
\definecolor{mPurple}{rgb}{0.58,0,0.82}
\definecolor{backgroundColour}{rgb}{0.95,0.95,0.92}

\lstdefinestyle{CStyle}{
    backgroundcolor=\color{backgroundColour},   
    commentstyle=\color{mGreen},
    keywordstyle=\textbf{\color{black}},
    numberstyle=\tiny\color{mGray},
    stringstyle=\color{mPurple},
    basicstyle=\footnotesize,
    breakatwhitespace=false,         
    breaklines=true,                 
    captionpos=b,                    
    keepspaces=true,                 
    numbers=left,                    
    numbersep=5pt,                  
    showspaces=false,                
    showstringspaces=false,
    showtabs=false,                  
    tabsize=2,
    frame=single,
    escapeinside={(*}{*)},
    language=C
}

\makeatletter
% This command ignores the optional argument for itemize and enumerate lists
\newcommand{\inlineitem}[1][]{%
\ifnum\enit@type=\tw@
    {\descriptionlabel{#1}}
  \hspace{\labelsep}
\else
  \ifnum\enit@type=\z@
       \refstepcounter{\@listctr}\fi
    \quad\@itemlabel\hspace{\labelsep}
\fi}
\makeatother

\newcommand{\onestaritem}{\refstepcounter{enumi}\item[$*$\theenumi.]}
\newcommand{\twostaritem}{\refstepcounter{enumi}\item[$**$\theenumi.]}
\DeclareMathOperator*{\maxi}{max}

\title{Lista 12: Fundamentos Estatísticos para Ciência dos Dados}
\author{Ricardo Pagoto Marinho}

\begin{document}
\maketitle
Fazer os exercícios 1,2,3,4,9,10,11,12,13,14,15,16,17,18,19,20,29 e 30 do capítulo 9.
	\begin{itemize}
		\item 1
		
		\begin{lstlisting}
aptos=read.table("aptosBH.txt",header = T)
attach(aptos)
par(mfrow=c(2,2))
plot(area,preco)
plot(quartos,preco)
plot(suites,preco)
plot(vagas,preco)
x = as.matrix(cbind(1, aptos[,2]))
b.simples = (solve(t(x) %*% x)) %*% (t(x) %*% preco)
b.simples
           [,1]
[1,] 200514.979
[2,]   3548.856
x = as.matrix(cbind(1, aptos[,2:5])) # matriz de desenho n x 2
b.all = (solve(t(x) %*% x)) %*% (t(x) %*% preco)
b.all
               [,1]
1       -269382.128
area       1915.898
quartos   59637.006
suites   111743.835
vagas    191404.127
		\end{lstlisting}
		
	\end{itemize}
\end{document}