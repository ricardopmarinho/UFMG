\documentclass[11pt,a4paper]{book}
\usepackage[brazilian]{babel}
\usepackage[utf8]{inputenc}
\usepackage[T1]{fontenc}
\usepackage[inline]{enumitem}
\usepackage{xcolor}
\usepackage{listings}
\usepackage{graphicx}
\usepackage{multicol}
\usepackage{amsmath}

\definecolor{mGreen}{rgb}{0,0.6,0}
\definecolor{mGray}{rgb}{0.5,0.5,0.5}
\definecolor{mPurple}{rgb}{0.58,0,0.82}
\definecolor{backgroundColour}{rgb}{0.95,0.95,0.92}

\lstdefinestyle{CStyle}{
    backgroundcolor=\color{backgroundColour},   
    commentstyle=\color{mGreen},
    keywordstyle=\textbf{\color{black}},
    numberstyle=\tiny\color{mGray},
    stringstyle=\color{mPurple},
    basicstyle=\footnotesize,
    breakatwhitespace=false,         
    breaklines=true,                 
    captionpos=b,                    
    keepspaces=true,                 
    numbers=left,                    
    numbersep=5pt,                  
    showspaces=false,                
    showstringspaces=false,
    showtabs=false,                  
    tabsize=2,
    frame=single,
    escapeinside={(*}{*)},
    language=C
}

\makeatletter
% This command ignores the optional argument for itemize and enumerate lists
\newcommand{\inlineitem}[1][]{%
\ifnum\enit@type=\tw@
    {\descriptionlabel{#1}}
  \hspace{\labelsep}
\else
  \ifnum\enit@type=\z@
       \refstepcounter{\@listctr}\fi
    \quad\@itemlabel\hspace{\labelsep}
\fi}
\makeatother

\newcommand{\onestaritem}{\refstepcounter{enumi}\item[$*$\theenumi.]}
\newcommand{\twostaritem}{\refstepcounter{enumi}\item[$**$\theenumi.]}

\title{Notas de aula: Fundamentos Estatísticos para Ciência dos Dados}
\author{Ricardo Pagoto Marinho}

\begin{document}
\maketitle
	\begin{itemize}
		\item 13/03
		
		$P(\cup A_i)~\leq~\sum~P(A_i)~\rightarrow$ é igual quantos os $A_is$ forem disjuntos.
		
		\item 15/03
		
		$P(A|B)~=~P(B)~\rightarrow$ quando A ocorre e não tem nenhuma influência sobre $B_0$.
		
		\item 20/03
		
		Variável aleatória: Lista de valores possíveis e lista de probabilidades associadas
		
		$\omega$ dentro de um $\Omega$.
		Exemplo: $\Omega$ = todos e-mails enviados.
		\begin{itemize}
			\item $\omega_0$ = é spam?
			\item $\omega_1$ = número de caracteres.
			\item $\cdots$
		\end{itemize}
		Elementos em uma mesma linha ($\omega_n$), são correlacionados. 
		
		\begin{itemize}
			\item Atribuir valores de probabilidades a uma V.A. $\rightarrow$ contar quantos elementos no $\Omega$ possuem aquela característica.
			
			$P(X=3) = P(A)$ onde A=$\lbrace \omega \in \Omega / \omega~tem~3~caras \rbrace$ em $\Omega$ = lançamento de 6 moedas.
			\item Esperança matemática $E(X)$
			
			$E(X)= \sum_i x_ip(xi) \approx \sum_i xi\times \frac{N_i}{N}$
			
			\item Distribuição Binomial
			
			$P(X=0) = (1-\theta)^{n}$
			
			$[X=0]=\lbrace\omega \in \Omega:X(\omega)=1\rbrace=\lbrace\omega \in \Omega: \omega \in \lbrace(\neg c,\neg c,\neg c,\cdots,\neg c)\rbrace\rbrace=P(\neg c~no~1º)\times P(\neg~c~no~2º)\times \cdots = (1-\theta)\times(1-\theta)\cdots = (1-\theta)^{n}$
			
		\end{itemize}
	\end{itemize}
\end{document}