\documentclass[11pt,a4paper]{book}
\usepackage[brazilian]{babel}
\usepackage[utf8]{inputenc}
\usepackage[T1]{fontenc}
\usepackage[inline]{enumitem}
\usepackage{xcolor}
\usepackage{listings}
\usepackage{graphicx}
\usepackage{multicol}
\usepackage{amsmath}

\definecolor{mGreen}{rgb}{0,0.6,0}
\definecolor{mGray}{rgb}{0.5,0.5,0.5}
\definecolor{mPurple}{rgb}{0.58,0,0.82}
\definecolor{backgroundColour}{rgb}{0.95,0.95,0.92}

\lstdefinestyle{CStyle}{
    backgroundcolor=\color{backgroundColour},   
    commentstyle=\color{mGreen},
    keywordstyle=\textbf{\color{black}},
    numberstyle=\tiny\color{mGray},
    stringstyle=\color{mPurple},
    basicstyle=\footnotesize,
    breakatwhitespace=false,         
    breaklines=true,                 
    captionpos=b,                    
    keepspaces=true,                 
    numbers=left,                    
    numbersep=5pt,                  
    showspaces=false,                
    showstringspaces=false,
    showtabs=false,                  
    tabsize=2,
    frame=single,
    escapeinside={(*}{*)},
    language=C
}

\makeatletter
% This command ignores the optional argument for itemize and enumerate lists
\newcommand{\inlineitem}[1][]{%
\ifnum\enit@type=\tw@
    {\descriptionlabel{#1}}
  \hspace{\labelsep}
\else
  \ifnum\enit@type=\z@
       \refstepcounter{\@listctr}\fi
    \quad\@itemlabel\hspace{\labelsep}
\fi}
\makeatother

\newcommand{\onestaritem}{\refstepcounter{enumi}\item[$*$\theenumi.]}
\newcommand{\twostaritem}{\refstepcounter{enumi}\item[$**$\theenumi.]}

\title{Resumo: Métodos Qualitativos de Pesquisa}
\author{Ricardo Pagoto Marinho}

\begin{document}
\maketitle
	\begin{itemize}
		\item Diferenças entre os paradigmas qualitativos e quantitativos
		
		\begin{tabular}{|p{3cm}|p{4cm}|p{4cm}|}
		\hline
		Questão & Qualitativo & Quantitativo\\
		\hline
		Característica do fenômeno investigado & Irreplicáveis, complexos, imprevisíveis e relativos a um contexto de ocorrência, impossível de isolar todas as variáveis & prever o comportamento dos fenômenos, conhecimento e controle das variáveis, replicável\\
		\hline
		Tipo de Problema & Exploração de problemas e contexto a respeito dos quais se busca respostas novas e imprevisíveis & Hipotetizável, \textit{i.e.}, pode-se formular e testar hipóteses a respeito da ocorrência\\
		\hline
		Raciocínio & Interpretativo-indutivo & Hipotético-dedutivo\\
		\hline
		Ação do investigador & Exploração de significados e análise de conteúdo e discurso & Manipulação de variáveis e análise estatística\\
		\hline
		Postura do investigador & Envolvimento interpretativo; análise de impactos éticos & Concepção de neutralidade\\
		\hline
		Tipo de resultados & Em profundidade & Abrangentes\\
		\hline
		\end{tabular}
		
		\item Os métodos qualitativos são procedimentos sistemáticos criados para investigar um dado fenômeno que não se presta à elaboração de hipóteses.
		
		\item Utilizamos para estudar objetos complexos e multifacetados, cujas ações dificilmente serão capturadas utilizando um conjunto limitado de variáveis.
		
		\item Características:
			\begin{itemize}
				\item Busca-se coletar o dado o mais próximo possível do contexto natural de ocorrência do fenômeno;
				\item Aspectos observáveis (opiniões, sentimentos, críticas, etc.) são coletados;
				\item Investigação em profundidade com pequenas amostras;
				\item Investigação de significados a partir da perspectiva do participante: a linguagem, o modo com interagem com a tecnologia, expressões faciais, fala...
				\item Análise sistemática, iterativa e interpretativa dos dados coletados: relacionar explicitamente os elementos coletados e os significados atribuídos através de uma atividade intelectual interpretativa.
			\end{itemize}			 
		\item Triangulação: utilizar diferentes métodos para uma melhor compreensão do fenômeno observado
		
		\item Paradigma de pesquisa
		
			\begin{itemize}
				\item Desenvolver pesquisa exploratória: gerar conhecimento e explicações teóricas, identificar um conceito básico que permitisse relacionar, sistematizar, entender e explicar os fenômenos sociais em que eles estavam interessados.
				
				\item Análise de processos sociais: concentra em experiências pessoais sobre um determinado fenômeno, analisa exemplos em que se vive com as consequências ou se lida com os termos gerais.
				
				\item Analisar a aplicação da teoria na prática: como conceitos teóricos eram adotados por grupos de profissionais e como diferentes grupos diferiam nessa aplicação 
			\end{itemize}
	\end{itemize}
\end{document}