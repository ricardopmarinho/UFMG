\documentclass[11pt,a4paper]{book}
\usepackage[brazilian]{babel}
\usepackage[utf8]{inputenc}
\usepackage[T1]{fontenc}
\usepackage[inline]{enumitem}
\usepackage{xcolor}
\usepackage{listings}
\usepackage{graphicx}
\usepackage{multicol}
\usepackage{amsmath}

\definecolor{mGreen}{rgb}{0,0.6,0}
\definecolor{mGray}{rgb}{0.5,0.5,0.5}
\definecolor{mPurple}{rgb}{0.58,0,0.82}
\definecolor{backgroundColour}{rgb}{0.95,0.95,0.92}

\lstdefinestyle{CStyle}{
    backgroundcolor=\color{backgroundColour},   
    commentstyle=\color{mGreen},
    keywordstyle=\textbf{\color{black}},
    numberstyle=\tiny\color{mGray},
    stringstyle=\color{mPurple},
    basicstyle=\footnotesize,
    breakatwhitespace=false,         
    breaklines=true,                 
    captionpos=b,                    
    keepspaces=true,                 
    numbers=left,                    
    numbersep=5pt,                  
    showspaces=false,                
    showstringspaces=false,
    showtabs=false,                  
    tabsize=2,
    frame=single,
    escapeinside={(*}{*)},
    language=C
}

\makeatletter
% This command ignores the optional argument for itemize and enumerate lists
\newcommand{\inlineitem}[1][]{%
\ifnum\enit@type=\tw@
    {\descriptionlabel{#1}}
  \hspace{\labelsep}
\else
  \ifnum\enit@type=\z@
       \refstepcounter{\@listctr}\fi
    \quad\@itemlabel\hspace{\labelsep}
\fi}
\makeatother

\newcommand{\onestaritem}{\refstepcounter{enumi}\item[$*$\theenumi.]}
\newcommand{\twostaritem}{\refstepcounter{enumi}\item[$**$\theenumi.]}

\title{Notas de aula: Métodos Qualitativos de Pesquisa}
\author{Ricardo Pagoto Marinho}

\begin{document}
\maketitle
	\begin{itemize}
		\item 16/04
		
		Condução de grupo focal: fazer as pessoas com personalidades diferentes falarem o mesmo tempo e não deixar o foco da conversa mudar muito.
		
		Anotações: Uma pessoa anota o nome de quem está falando na ordem para que na hora de gerar o áudio, ficar mais fácil.
		Logo após de acabar o grupo, registrar o que não pode ser escrito durante a conversa.
		
		\item 23/04
		
		Coleta de dados em contextos reais:
		
		\begin{itemize}
			\item observar contextos desconhecidos para aprender sobre ele e preparar uma entrevista.
			
			\item aspectos hierárquicos: se tem algum chefe para autorizar.
			
			produtividade: exemplo, não pode atrapalhar a produtividade da empresa
			
			\item Diários de uso: estudo entre laboratório e contexto real: quem faz o registro é o usuário, não o pesquisador.
			Ele faz o registro controlado (com as informações necessárias).
			
			
			\begin{tabular}{|p{4cm}|p{4cm}|}
			\hline
			vantagens & desvantagens\\
			\hline
			assim que a apessoa usa, ela faz o registro. a experiência ainda está fresca na memória & a pessoa não entende oque tipo de coisa ela tem que registrar (muito abstrado, não específico no nível que o entrevistador gostaria) $\rightarrow$ exemplos de entradas\\
			\hline
			\end{tabular}
			
			\item diário pode ser mais ou menos definido em relação às perguntas, normalmente mais estruturado
			
			\item diário de elicitação é menos definido quanto aos eventos de interesses
			
			\item análise qualitativa do discurso: motivos que levaram ao usuário tomar uma decisão
			
		\end{itemize}
		
	\item 02/05
	Codificação
	
	
	\begin{itemize}
		\item Código pre-existente: criado por algum outro trabalho. vindo da literatura não precisa ser necessariamente sobre o que você está fazendo, ou seja, pode vir de um framework, ontologia, etc.
		\item aberta: depende da questão de pesquisa, do interpretador. para validar, existe uma etapa para criar o código e utiliza triangulação para consolidar os resultados $\rightarrow$ mais de uma pessoa cria o código individualmente e entram em consenso sobre os códigos.
		\item registrar o código criado, o que significa cada palavra
		\item 
	\end{itemize}
		
		\item 07/05
		Análise temática
		
		\begin{itemize}
			\item Texto: transcrição dos discursos ou outros textos existentes.
			\item flexível: as coisas relevante surgem em processos iterativos que devem ser revistas
			\item descrição dos dados anaálise tem objetivo ser descritiva sobre como os participantes veem o problema quando a área é nova, quando não for, pode focar em uma tópico específico.
			\item análise teórico: usar análises anteriores para analisar os seus dados.
			\item análise semântica: ampla e fazem descrição da superfície, a latente foca porque foi feito naquele formato.
			\item durante a coleta de dados, podemos perceber pontos de interesses que surgiram na coleta.
		\end{itemize}
	\end{itemize}
\end{document}