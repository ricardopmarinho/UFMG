\documentclass[11pt,a4paper]{book}
\usepackage[brazilian]{babel}
\usepackage[utf8]{inputenc}
\usepackage[T1]{fontenc}
\usepackage[inline]{enumitem}
\usepackage{xcolor}
\usepackage{listings}
\usepackage{graphicx}
\usepackage{multicol}
\usepackage{amsmath}

\definecolor{mGreen}{rgb}{0,0.6,0}
\definecolor{mGray}{rgb}{0.5,0.5,0.5}
\definecolor{mPurple}{rgb}{0.58,0,0.82}
\definecolor{backgroundColour}{rgb}{0.95,0.95,0.92}

\lstdefinestyle{CStyle}{
    backgroundcolor=\color{backgroundColour},   
    commentstyle=\color{mGreen},
    keywordstyle=\textbf{\color{black}},
    numberstyle=\tiny\color{mGray},
    stringstyle=\color{mPurple},
    basicstyle=\footnotesize,
    breakatwhitespace=false,         
    breaklines=true,                 
    captionpos=b,                    
    keepspaces=true,                 
    numbers=left,                    
    numbersep=5pt,                  
    showspaces=false,                
    showstringspaces=false,
    showtabs=false,                  
    tabsize=2,
    frame=single,
    escapeinside={(*}{*)},
    language=C
}

\makeatletter
% This command ignores the optional argument for itemize and enumerate lists
\newcommand{\inlineitem}[1][]{%
\ifnum\enit@type=\tw@
    {\descriptionlabel{#1}}
  \hspace{\labelsep}
\else
  \ifnum\enit@type=\z@
       \refstepcounter{\@listctr}\fi
    \quad\@itemlabel\hspace{\labelsep}
\fi}
\makeatother

\newcommand{\onestaritem}{\refstepcounter{enumi}\item[$*$\theenumi.]}
\newcommand{\twostaritem}{\refstepcounter{enumi}\item[$**$\theenumi.]}

\title{Lista 6: Fundamentos Estatísticos para Ciência dos Dados}
\author{Ricardo Pagoto Marinho}

\begin{document}
\maketitle
	\begin{itemize}
		\item 1)
		A tabela a seguir é dada no problema:
		
		\begin{tabular}{c c c c c c c c c c c}
		\hline
		k&0&1&2&3&4&5&6&7&8&9\\
		Obs&60&62&67&68&64&56&62&44&58&67\\
		Esp&??&??&??&??&??&??&??&??&??&??\\
		\hline
		\end{tabular}				
		
		Com ela, temos que preencher a última linha, ou seja, a da quantidade esperada de cada dígito na expansão decimal do número $\pi$.
		Como foi dito que os dígitos são escolhidos de forma aleatória, cada um terá a probabilidade de 10\% de aparecer, ou seja, é esperado que cada dígito apareça 60.8 vezes na expansão.
		Logo, a última linda da tabela é completada como segue:
		
		\begin{tabular}{c c c c c c c c c c c}
		\hline
		k&0&1&2&3&4&5&6&7&8&9\\
		Obs&60&62&67&68&64&56&62&44&58&67\\
		Esp&60.8&60.8&60.8&60.8&60.8&60.8&60.8&60.8&60.8&60.8\\
		\hline
		\end{tabular}
		
		Fazendo os cálculos para o teste qui-quadrado em R, foi obtido o seguinte resultado:
		\begin{lstlisting}
obs<-c(60,62,67,68,64,56,62,44,58,67)

esp<-rep(60.8,10)

qquad<-sum((obs-esp)^2/esp)
		
qquad

7.493421		
		\end{lstlisting}
		Para rejeitar ou aceitar essa distribuição, devemos olhar a tabela do teste com Grau de Liberdade 9, já que possuímos 10 classes (os 10 dígitos).
		Com base na seguinte tabela\\(encontrada em http://www.cultura.ufpa.br/dicas/biome/biotaqui.htm):
		
		\begin{tabular}{c c}
		\hline
		G.L.&$\chi^2$\\
		1&3.841\\
		2&5.991\\
		3&7.815\\
		4&9.488\\
		5&11.070\\
		6&12.592\\
		7&14.067\\
		8&15.507\\
		9&16.919\\
		\hline
		\end{tabular}
		
		é possível observar que para um Grau de Liberdade 9, devemos rejeitar a hipótese caso o valor do teste seja maior do que 16.919.
		Como o resultado deu 7.493421, podemos admitir que a hipótese na qual os 608 primeiro dígitos do número $\pi$ sigam uma distribuição uniforme.
		
		\item 2)
		
		x<-rnorm(50,5,3)

		ks.test(x,"pnorm",5,3)
\\
		One-sample Kolmogorov-Smirnov test

		data:  x

		D = 0.12562, p-value = 0.3777

		alternative hypothesis: two-sided
		
		sqrt(50)*0.12562
		
		0.8882675
		
		Como $\sqrt{n}D_n<1.36$, não podemos ter rejeitar os dados com certeza.
		Agora utilizando $\mu =5$ e $\sigma=2$, é esperado que os dados tenham que ser rejeitados, já que é outra distribuição.
		Os resultados obtidos são os seguintes:
		
		ks.test(x,"pnorm",5,2)

		One-sample Kolmogorov-Smirnov test
\\
		data:  x

		D = 0.19948, p-value = 0.03211
		
		alternative hypothesis: two-sided

		sqrt(50)*0.19948
		
		1.410537
		
		Agora, $\sqrt{n}D_n>1.36$, como esperado, os dados deverão ser rejeitados.
		
		\item 3)
		
		\begin{lstlisting}
# funcao para calcular o teste de Kolmogorov:
# 	Nome:kol
#	Entradas: vec-> vetor com os valores a serem testados
#			  mu-> media utilizada para o teste
#			  sigma-> desvio padrao utilizado para o
#					 teste
#	femp-> vetor de valores empiricos a partir de 'vec'
#		   femp=#{x_i<=x}/n
#	xc-> vetor para calculo da cdf da distribuicao criada
#		dentro da funcao
#	x->vetor de valores de uma distribuicao normal criado 
#	  a partir das entradas mu e sigma
kol=function(vec,mu=mean(vec),sigma=sqrt(var(vec))){
	femp<-0 
   	xc<-0
	j<-1
	x<-rnorm(length(vec),mu,sigma)
	x<-sort(x)
	for(i in x){
		femp[j]<-sum(vec<=x[j])/length(vec)
		xc[j]<-sum(x<=x[j])/length(vec)
		j<-j+1
	}
	f<-abs(femp-xc)
	return(sqrt(length(vec))*max(f))
}
		\end{lstlisting}
		
		\item 4)
		
		\begin{itemize}
			\item $\mathbb{F}_Y(0.9)\approx 0$
			\item $\mathbb{F}_Y(1.1)\approx 0.5$
			\item $\mathbb{F}_Y(1.8)\approx 0.95$
			\item $\mathbb{F}_Y(2.1)\approx 1$
		\end{itemize}
		
		\item 10)
		
		\begin{itemize}
			\item Falso
			\item Verdadeiro
			\item Falso
			\item Falso
			\item Verdadeiro
			\item Falso
		\end{itemize}
	\end{itemize}
\end{document}