\documentclass[11pt,a4paper]{book}
\usepackage[brazilian]{babel}
\usepackage[utf8]{inputenc}
\usepackage[T1]{fontenc}
\usepackage[inline]{enumitem}
\usepackage{xcolor}
\usepackage{listings}
\usepackage{graphicx}
\usepackage{multicol}
\usepackage{amsmath}

\definecolor{mGreen}{rgb}{0,0.6,0}
\definecolor{mGray}{rgb}{0.5,0.5,0.5}
\definecolor{mPurple}{rgb}{0.58,0,0.82}
\definecolor{backgroundColour}{rgb}{0.95,0.95,0.92}

\lstdefinestyle{CStyle}{
    backgroundcolor=\color{backgroundColour},   
    commentstyle=\color{mGreen},
    keywordstyle=\textbf{\color{black}},
    numberstyle=\tiny\color{mGray},
    stringstyle=\color{mPurple},
    basicstyle=\footnotesize,
    breakatwhitespace=false,         
    breaklines=true,                 
    captionpos=b,                    
    keepspaces=true,                 
    numbers=left,                    
    numbersep=5pt,                  
    showspaces=false,                
    showstringspaces=false,
    showtabs=false,                  
    tabsize=2,
    frame=single,
    escapeinside={(*}{*)},
    language=C
}

\makeatletter
% This command ignores the optional argument for itemize and enumerate lists
\newcommand{\inlineitem}[1][]{%
\ifnum\enit@type=\tw@
    {\descriptionlabel{#1}}
  \hspace{\labelsep}
\else
  \ifnum\enit@type=\z@
       \refstepcounter{\@listctr}\fi
    \quad\@itemlabel\hspace{\labelsep}
\fi}
\makeatother

\newcommand{\onestaritem}{\refstepcounter{enumi}\item[$*$\theenumi.]}
\newcommand{\twostaritem}{\refstepcounter{enumi}\item[$**$\theenumi.]}

\title{Notas de aula: Fundamentos Estatísticos para Ciência dos Dados}
\author{Ricardo Pagoto Marinho}

\begin{document}
\maketitle
	\begin{itemize}
		\item 13/03
		
		$P(\cup A_i)~\leq~\sum~P(A_i)~\rightarrow$ é igual quantos os $A_is$ forem disjuntos.
		
		\item 15/03
		
		$P(A|B)~=~P(B)~\rightarrow$ quando A ocorre e não tem nenhuma influência sobre $B_0$.
		
		\item 20/03
		
		Variável aleatória: Lista de valores possíveis e lista de probabilidades associadas
		
		$\omega$ dentro de um $\Omega$.
		Exemplo: $\Omega$ = todos e-mails enviados.
		\begin{itemize}
			\item $\omega_0$ = é spam?
			\item $\omega_1$ = número de caracteres.
			\item $\cdots$
		\end{itemize}
		Elementos em uma mesma linha ($\omega_n$), são correlacionados. 
		
		\begin{itemize}
			\item Atribuir valores de probabilidades a uma V.A. $\rightarrow$ contar quantos elementos no $\Omega$ possuem aquela característica.
			
			$P(X=3) = P(A)$ onde A=$\lbrace \omega \in \Omega / \omega~tem~3~caras \rbrace$ em $\Omega$ = lançamento de 6 moedas.
			\item Esperança matemática $E(X)$
			
			$E(X)= \sum_i x_ip(xi) \approx \sum_i xi\times \frac{N_i}{N}$
			
			\item Distribuição Binomial
			
			$P(X=0) = (1-\theta)^{n}$
			
			$[X=0]=\lbrace\omega \in \Omega:X(\omega)=1\rbrace=\lbrace\omega \in \Omega: \omega \in \lbrace(\neg c,\neg c,\neg c,\cdots,\neg c)\rbrace\rbrace=P(\neg c~no~1º)\times P(\neg~c~no~2º)\times \cdots = (1-\theta)\times(1-\theta)\cdots = (1-\theta)^{n}$
			
		\end{itemize}
		
		\item 27/03
		
		$P(Y \in (y_0\pm \frac{\delta}{2}))~=~\int_{y_0-\frac{\delta}{2}}^{y_0+\frac{\delta}{2}}f^{*}(y)dy\approx~f^{*}(y_0)2\times\frac{\delta}{2}=f^{*}(y_0)\times\delta$
		
		Teste de Kolmogorov:
		
		$\sqrt{n}(D_n)\rightarrow~K$, onde K é uma Variável Aleatória contínua.
		
		Se o modelo é o verdadeiro, quando comparado com os dados, a distância entre eles multiplicado por $\sqrt{n}$ vai cair dentro da densidade de K.
		Se não cair, provavelmente seu modelo não é adequado.
		Quanto maior o número de dados, mais confiável o resultado.
		
		\item 03/04
		
		Variáveis aleatórias: Lista de valores possíveis + probabilidades associadas
		
		\begin{tabular}{|c|c|c|}
			\hline
			& Discretas & Contínuas\\
			\hline
			Valores & 0,1,2,...& [0,1] ou [0,$\infty$)\\
			\hline
			Probabilidades & $p_0,p_1,p_2,...$ & Densidade sob a curva\\
			\hline
		\end{tabular}
		
		\begin{tabular}{|c|c|}
		\hline
		& $E(X)$\\
		\hline
		Discreta & $\sum_i x_i\times~P(X=x_i)$\\
		\hline
		Contínua & $\int x\times~f(x)dx$\\
		\hline
		\end{tabular}
		
		Teste qui-quadrado:
		Compara o modelo com os dados.
		Serve para dados contínuos e discretos.
		
		pchisq(15,4)-pchisq(10,4) $\rightarrow $ comando em R para saber o valor do teste qui-quadrado no intervalo [10,15] com 4 graus de liberdade.
		
		pchisq(1.13,4) $\rightarrow$ probabilidade de uma distribuição qui-quadrado com 4 graus de liberdade ser menor do que 1.13.
		
		1-pchisq(1.13,4)=pvalors
		
		\item 05/04
		
		As variáveis são i.i.d. (independentes e identicamente distribuídas) se:
		\begin{itemize}
			\item elas forem todas independentes
			\item possuírem todas a mesma distribuição
		\end{itemize}
		
		\begin{itemize}
			\item Transformação de V.A.s
			
			X é V.A.
			
			Y=g(X) é V.a.
			
			g(X) é função matemática.
			
			Distribuição de Y? 
			\begin{itemize}
				\item inverter g : Obter $F_y(y)=P(Y \leq~y)$ e deriva para obter a densidade $f_y(Y)=F'(y)$
				\item Y=g(X) e $X=g^{-1}(Y)=h(Y)$
				
				então $f_y(y)=f_x(h(y)).|h'(y)|$
				
				Exemplo:$ f_x(x) = \left.
				\begin{cases}
					0, & \mbox{se } x \ni (0,1)\\
					1, & \mbox{se } x \in (0,1)
				\end{cases}
				\right.
				$
				
				$Y=X^2\rightarrow X=\sqrt{Y}=h(Y)$
				
				Então:
				
				$f_y(y)=f_x(\sqrt{y})\times |\frac{d\sqrt{y}}{dy}$
				
				Se quisermos $E(Y) = E(g(x)$
				\begin{enumerate}
					\item $E(Y) = \int y~f_Y(y)dy$ ($f_Y(y)$ é obtida de uma das duas maneiras anteriores)
					\item $=\int_{-\infty}^{\infty} g(x)\times~f_x(x)dx$
				\end{enumerate}
			\end{itemize} 
			
		\end{itemize}
	\end{itemize}
\end{document}