\documentclass[11pt,a4paper]{article}
\usepackage[brazilian]{babel}
\usepackage[utf8]{inputenc}
\usepackage[T1]{fontenc}
\usepackage[inline]{enumitem}
\usepackage{xcolor}
\usepackage{listings}
\usepackage{graphicx}
\usepackage{multicol}
\usepackage{amsmath}
\usepackage{amssymb}

\definecolor{mGreen}{rgb}{0,0.6,0}
\definecolor{mGray}{rgb}{0.5,0.5,0.5}
\definecolor{mPurple}{rgb}{0.58,0,0.82}
\definecolor{backgroundColour}{rgb}{0.95,0.95,0.92}

\lstdefinestyle{CStyle}{
    backgroundcolor=\color{backgroundColour},   
    commentstyle=\color{mGreen},
    keywordstyle=\textbf{\color{black}},
    numberstyle=\tiny\color{mGray},
    stringstyle=\color{mPurple},
    basicstyle=\footnotesize,
    breakatwhitespace=false,         
    breaklines=true,                 
    captionpos=b,                    
    keepspaces=true,                 
    numbers=left,                    
    numbersep=5pt,                  
    showspaces=false,                
    showstringspaces=false,
    showtabs=false,                  
    tabsize=2,
    frame=single,
    escapeinside={(*}{*)},
    language=C
}

\makeatletter
% This command ignores the optional argument for itemize and enumerate lists
\newcommand{\inlineitem}[1][]{%
\ifnum\enit@type=\tw@
    {\descriptionlabel{#1}}
  \hspace{\labelsep}
\else
  \ifnum\enit@type=\z@
       \refstepcounter{\@listctr}\fi
    \quad\@itemlabel\hspace{\labelsep}
\fi}
\makeatother

\newcommand{\onestaritem}{\refstepcounter{enumi}\item[$*$\theenumi.]}
\newcommand{\twostaritem}{\refstepcounter{enumi}\item[$**$\theenumi.]}

\title{Resumo: Fundamentos Estatísticos para Ciência dos Dados}
\author{Ricardo Pagoto Marinho}

\begin{document}
\maketitle
	\section{Regra de Bayes}
	Inverte as probabilidades de interesse.
	
	Exemplo:
	
	$\mathbb{P}(A|B) = \frac{\mathbb{P}(B|A)\mathbb{P}(A)}{\mathbb{P}(B)}$
	
	\section{Função distribuição acumulada de probabilidade}
	$\mathbb{F}(x)$ definida $\forall~x~\in~\mathbb{R}$ é dada por:
	\begin{align*}
		\mathbb{F}: & \mathbb{R}\rightarrow~[0,1]\\
		& x\rightarrow~\mathbb{F}(x)=\mathbb{P}(X\leq~x)
	\end{align*}
		
	Caso geral de $\mathbb{F}(x)$
	\begin{align*}
		\mathbb{F}(x)=\mathbb{P}(X\leq~x)=\sum_{x_i\leq~x}p(x_i)
	\end{align*}
		
	\section{Esperança matemática ($\mathbb{E}(X)$)}
		
		\subsection{V.A. Discreta}
			O valor esperado de uma V.A. discreta é a soma de seus valores possíveis ponderados pelas suas probabilidades respectivas.
			\begin{align*}
				\mathbb{E}(X)=\sum_i~x_ip(x_i)
			\end{align*}
	
			Suponha que numa amostra grande de instâncias, $x_i$ apareceu $N_i$ vezes.
			A probabilidade de $x_i$ ocorrer na amostra é sua frequência relativa, \textit{i.e.}:
		
			\begin{align*}
				p(x_i)=\mathbb{P}(X=x_i)\approx~\frac{N_i}{N}
			\end{align*}
		
			Logo:
			\begin{align*}
				\mathbb{E}(X)=\sum_i~x_ip(x_i)\approx \sum_i~x_i\frac{N_i}{N}
			\end{align*}
		
			Se a amostra for grande, o número teórico $\mathbb{E}(X)$ é aproximadamente igual à média aritmética dos N elementos da amostra.
	
		\subsection{V.A. contínuas}
	
		\begin{eqnarray*}
			\mathbb{E}(Y)=\int_{-\infty}^{\infty} yf(y)dy
		\end{eqnarray*}
		
		\subsection{Linearidade da esperança}
		
		Caso geral: Y=a+bX, onde a e b são constantes.
		Então $\mathbb{E}(X)$ e $\mathbb{E}(Y)$ estão relacionadas:
		
		\begin{eqnarray*}
			\mathbb{E}(Y)=\mathbb{E}(a+bX)=a+b\mathbb{E}(X)
		\end{eqnarray*}
		
		Exemplo:
		
		Medimos uma temperatura aleatória \textbf{C} em graus Celsius.
		Suponha que $\mathbb{E}(C)=28$ graus.
		Seja F a V.A. que mede a temperatura em graus Fahrenheit.
		Temos que C e F estão relacionadas por: $F=32+\frac{9}{5}C$.
		Pelo caso geral da linearidade, a=32 e b=$\frac{9}{5}$.
		Logo
		
		\begin{eqnarray*}
			\mathbb{E}(F)=&\mathbb{E}(32+\frac{9}{5}C)\\
			=&32+\frac{9}{5}\mathbb{E}(C)\\
			=&32+\frac{9}{5}\times 28\\
			=&82.4
		\end{eqnarray*}
		
		Caso duas variáveis aleatórias sejam DISJUNTAS:
		\begin{eqnarray*}
			\mathbb{E}(X+Y)=\mathbb{E}(X)+\mathbb{E}(Y)
		\end{eqnarray*}
		
		Se independentes:
		
		\begin{eqnarray*}
			\mathbb{E}(XY)=\mathbb{E}(X)\mathbb{E}(Y)
		\end{eqnarray*}
		
		\section{Variância}
		
		\begin{eqnarray*}
			\mathbb{V}(X)=\mathbb{E}((X-\mu)^2)
		\end{eqnarray*}

		Outra fórmula:
		
		\begin{eqnarray*}
			\mathbb{V}(X)=&\mathbb{E}((X-\mu)^2)\\
			=&\mathbb{E}(X^2)-(\mu)^2\\
			=&\mathbb{E}(X^2)-(\mathbb{E}(X))^2
		\end{eqnarray*}
		
		Seja X uma v.a. com $\mu_x=\mathbb{E}(X)$ e $\sigma^2=\mathbb{V}(X)$.
		Se Y=a+bX, então $\mu_y=\mathbb{E}(Y)=a+b\mu_x$ e
		
		\begin{eqnarray*}
			\sigma_y^2=\mathbb{V}(Y)=\mathbb{V}(a+bX)=b^2\mathbb{V}(X)=b^2\sigma_x^2
		\end{eqnarray*}
		
		Em termos de DP das v.a.'s:
		
		\begin{eqnarray*}
			DP_y=|b|DP_X
		\end{eqnarray*}
		
		Se as v.a.'s são independentes, temos:
		
		\begin{eqnarray*}
			\mathbb{V}(X+Y)=\mathbb{V}(X)+\mathbb{V}(Y)
		\end{eqnarray*}
		
		\subsection{Caso discreto}
		
		Como $\mathbb{E}(g(X))=\sum_ig(x_i)\mathbb{P}(X=x_i)$, e tomando $g(X)=(X-\mu)^2$, então:
		
		\begin{eqnarray*}
			\mathbb{V}(X)=E((X-\mu)^2)=\sum_i(x_i-\mu)^2\mathbb{P}(X=x_i)
		\end{eqnarray*}
		
		\subsection{Caso contínuo}
		\begin{eqnarray*}
			\mathbb{V}(X)=\mathbb{E}((X-\mu)^2)=\int(x-\mu)^2f(x)dx
		\end{eqnarray*}
		
		\section{Distribuição de Bernoulli}
		
		É a distribuição mais simples: dois resultados possíveis
		$X(\omega)~\in~\lbrace0,1\rbrace~\forall~\omega~\in~\Omega$
		
		Duas probabilidades são definidas:
		\begin{itemize}
			\item p(0)=$\mathbb{P}(X=0)=\mathbb{P}(\omega~\in~\Omega:X(\omega)=0)$
			\item p(1)=$\mathbb{P}(X=1)=\mathbb{P}(\omega~\in~\Omega:X(\omega)=1)$
		\end{itemize}
		
		$p(0)+p(1)=1\rightarrow~p(1)=1-p(0)$
		
		É comum escrever $p(1)=p$ e $p(0)=q$.
		Daí, $\mathbb{E}(X)=1\times p+0\times(1-p)=p$
		
		Como a média aritmética dessa distribuição é a proporção de 1's na amostra:
		\begin{align*}
			\mathbb{E}(X)\approx\hat{p}=\frac{1}{N}\sum_i~x_i
		\end{align*}
		
		\section{Distribuição Binomial}
		
		Frequentemente utilizada quando um número máximo possível grande de $n$ de repetições e $\theta$ muito pequeno.
		
		$n$ repetições independentes de um experimento de Bernoulli: sucesso ou fracasso.
		Probabilidade de sucesso é igual a $\theta\in[0,1]$ 
		
		A V.A. $X$ conta o número total de sucessos: $X\sim~Bin(n,\theta)$.
		Os valores possíveis são: $0,1,2,\cdots,n$ e suas probabilidades, respectivamente são: $(1-\theta)^n,~n\theta(1-\theta),\cdots,\theta^n$
		
		Exemplo: $n$ lançamentos de uma moeda não viciada.
		\begin{align*}
			Cara&\rightarrow~C\\
			Coroa&\rightarrow~\tilde{C}
		\end{align*}
		$P(X=0) = (1-\theta)^{n}$
			
		$[X=0]=\lbrace\omega \in \Omega:X(\omega)=0\rbrace=\lbrace\omega \in \Omega: \omega \in \lbrace(\tilde{C},\tilde{C},\tilde{C},\cdots,\tilde{C})\rbrace\rbrace=P(\tilde{C}~no~1º)\times P(\tilde{C}~no~2º)\times \cdots = (1-\theta)\times(1-\theta)\cdots = (1-\theta)^{n}$
		
		\begin{itemize}
			\item Fórmula geral:
			\begin{align*}
				\mathbb{P}(Y=k)=\frac{n!}{k!(n-k)!}\theta^k(1-\theta)^{n-k}
			\end{align*}
			\item $\mathbb{E}(Y)=n\theta$ e DP=$\sqrt{\mathbb{V}(Y)}=\sqrt{n\theta(1-\theta)}$
		\end{itemize}
		
		\section{Distribuição Multinomial}
		
		Mais de duas categorias de resultados nos experimentos, diferente da Binomial que são só duas (1 ou 0).
		Ao fim de $n$ lançamentos, teremos um vetor aleatório multinomial que conta quantas vezes cada categoria apareceu no experimento.
		Temos $k$ categorias:
		\begin{align*}
			(N_1,N_2,\cdots,N_k)\sim~\mathbb{M}(n;\theta_1,\cdots,\theta_k)
		\end{align*}
		Sendo que $\theta_1,\cdots,\theta_k$ são as probabilidades de cada categoria.
		
		Exemplo: lançamento de um dado.
		$k=6$
		\begin{align*}
			N_1&=n^o~de~lançamentos~na~categoria~1\\
			N_2&=n^o~de~lançamentos~na~categoria~2\\
			N_3&=n^o~de~lançamentos~na~categoria~3\\
			&\vdots \\
			N_6&=n^o~de~lançamentos~na~categoria~6\\
		\end{align*}
		\begin{align*}
			(N_1,N_2,\cdots,N_6)\sim~\mathbb{M}(n;\theta_1,\cdots,\theta_6)
		\end{align*}
		
		Podemos escrever a Binomial como uma Multinomial de duas categorias: sucesso e fracasso.
		X é o número de sucessos em $n$ repetições.
		
		\begin{align*}
			(X,n-X)\sim~\mathbb{M}(n;\theta,1-\theta)
		\end{align*}
		
		A probabilidade de ocorrer uma configuração do vetor aleatório é:
		
		\begin{align}
			\mathbb{P}(\textbf{N}=(n_1,n_2,\cdots,n_k))=\frac{n!}{n_1!n_2!\cdots n_k!}\theta_{1}^{n_1}\theta_{2}^{n_2}\cdots\theta_{k}^{n_k}
			\label{eq:eq1}
		\end{align}
		
		Exemplo:
		8 lançamentos de um dado.
		A probabilidade de:
		\begin{align*}
			\mathbb{P}(\textbf{N}=(2,0,2,1,0,3))
		\end{align*}
		
		Existem várias configurações de $\omega$ as quais 8 lançamentos levam ao resultado acima.
		Uma é $\omega=(3,1,6,6,1,4,6,3)$.
		Logo:
		\begin{align*}
			\textbf{N}(\omega)=(N_1(\omega),N_2(\omega),\cdots,N_6(\omega))=(2,0,2,1,0,3)
		\end{align*}
		A probabilidade de sair essa configuração, levando em conta que os lançamentos são independentes é:
		\begin{eqnarray*}
			\mathbb{P}(\omega=(3,1,6,6,1,4,6,3))&=&\mathbb{P}(sair~3~no~1^o~E~sair~1~no~2^o~E\cdots~sair~3~no~8^o\\
			& = &\mathbb{P}(sair~3~no~1^o)\mathbb{P}(sair~1~no~2^o)\cdots\mathbb{P}(sair~3~no~8^o)\\
			& = &\theta_3~\theta_1~\theta_6~\theta_6~\theta_1~\theta_4~\theta_6~\theta_3\\
			& = &\theta_{1}^{2}~\theta_{2}^{0}~\theta_{3}^{2}~\theta_{4}^{1}~\theta_{5}^{0}~\theta_{6}^{3}
		\end{eqnarray*}
		
		Se a sequência $\omega$ tiver $n$ lançamentos:
		\begin{eqnarray*}
			n_1 & aparições~da~face 1\\
			n_2 & aparições~da~face 2\\
			&\vdots\\
			n_6 & aparições~da~face 6
		\end{eqnarray*}
		
		Teremos:
		\begin{eqnarray*}
			\mathbb{P}(\omega)=\theta_{1}^{n_1}~\theta_{2}^{n_2}~\theta_{3}^{n_3}~\theta_{4}^{n_4}~\theta_{5}^{n_5}~\theta_{6}^{n_6}
		\end{eqnarray*}
		
		Dessa forma, seja $A$ o evento formado por todos os $\omega$ tais que $\mathbb{P}(\textbf{N}=(2,0,2,1,0,3))$
		
		$\mathbb{P}(\textbf{N}=(2,0,2,1,0,3))=\mathbb{P}(A)=\sum_{\omega\in A}\mathbb{P}(\omega)=c\times\theta_{1}^{2}~\theta_{2}^{0}~\theta_{3}^{2}~\theta_{4}^{1}~\theta_{5}^{0}~\theta_{6}^{3}$
		Onde $c$ é o número de sequências de tamanho 8 tais que $\mathbb{P}(\textbf{N}=(2,0,2,1,0,3))$
		$c$ é o número de permutações do vetor $\omega=(3,1,6,6,1,4,6,3)$.
		Generalizando para k categorias, temos:
		\begin{eqnarray*}
			\textbf{N}=(N_1,N_2,\cdots,N_k)\sim\mathbb{M}(n;\theta_1,\cdots,\theta_n)
		\end{eqnarray*}
		
		Então, chegamos na Equação~\ref{eq:eq1}.
		
		\section{Distribuição de Poisson}
		
		Frequentemente utilizada em situações onde o número de ocorrências não tem um limite claro para o limite.
		
		$\mathbb{P}(Y=k)=\frac{\lambda^k}{k!}e^{-\lambda}$
		
		$\mathbb{E}(Y)=\lambda$
		
		\section{Distribuição geométrica}
		
		Y é o número de \textbf{fracassos} em uma sequência de ensaios independentes de Bernoulli até que um sucesso (probabilidade $\theta$) seja observado.
		Logo, Y=0 significa que no primeiro ensaio houve um sucesso e $\mathbb{P}(Y=0)=\mathbb{P}(S)=\theta$.
		Y=1 significa que o primeiro ensaio foi um fracasso e o segundo sucesso: $\mathbb{P}(Y=1)=\mathbb{P}(FS)=(1-\theta)\theta$.
		
		De forma geral, $\mathbb{P}(Y=k)=(1-\theta)^k\theta$
		
		Para uma geométrica com probabilidade de sucesso $\theta$:
		
		\begin{eqnarray*}
			\mathbb{E}(Y)=\frac{1}{\theta}
		\end{eqnarray*}
		
		Uma distribuição geométrica com $\theta$ alto significa que a probabilidade de sucesso é grande.
		Logo, a função de distribuição de probabilidade se concentrará mais nos números iniciais.
		
		\section{Distribuição de Zipf ou de Pareto}
		
		$X\in {1,2,3,\cdots,N}$, sendo que N pode ser infinito.
		
		\begin{eqnarray*}
			\mathbb{P}(X=k)=\frac{C}{k^(1+\alpha)},~com~\alpha > 0
		\end{eqnarray*}
		
		\textbf{C} é uma constante que garante que as probabilidades somem 1:
		
		\begin{eqnarray*}
			1=&\mathbb{P}(X=1)+\mathbb{P}(X=2)+\mathbb{P}(X=3)+\ldots\\
			=&C(\frac{1}{1^{1+\alpha}}+\frac{1}{2^{1+\alpha}}+\frac{1}{3^{1+\alpha}}+\ldots\\
			=&C\sum_{k=1}^N\frac{1}{k^{1+\alpha}}
		\end{eqnarray*}
		
		O que implica que:
		
		\begin{eqnarray*}
			C=\frac{1}{\sum_{k=1}^N\frac{1}{k^{1+\alpha}}}
		\end{eqnarray*}
		
		\textbf{IMPORTANTE:} 
		
		\begin{eqnarray*}
			\mathbb{P}(X=k)\propto\frac{1}{k^{1+\alpha}}
		\end{eqnarray*}
		
		\textit{i.e.}, inversamente proporcional a uma potência de k.
		
		Com $\alpha=1.0$, a probabilidade de 0 é maior ($\approx 0.6$).
		Com $\alpha=0.5$, a probabilidade de 0 diminui.
		
		Como $\mathbb{P}(Y=k)\propto\frac{1}{k}$:
		\begin{eqnarray*}
			\mathbb{P}(Y=1)\propto 1\\
			\mathbb{P}(Y=2)\propto\frac{1}{2}\\
			\mathbb{P}(Y=2)\propto\frac{1}{3}, etc
		\end{eqnarray*}
		
		\section{Desigualdade de Tchebyshev}
		
		\begin{eqnarray*}
			\mathbb{P}(|Y-\mu|>k\sigma)\leq \frac{1}{k^2}
		\end{eqnarray*}
		
		\section{Normal bivariada}
		
		Importante distribuição para um vetor aleatório: $\textbf{Y}=Y_1,Y_2$.
		Cada v.a. segue uma distribuição gaussiana com sua própria esperança $\mu_j$ e variância $\sigma_{j}^{2}$, \textit{i.e.}, $Y_1\sim N(\mu_1,\sigma_1^2)$ e $Y_2\sim N(\mu_2,\sigma_2^2)$
		As v.a.s não são (em geral) independentes, ou seja, $Y_2$ muda se soubermos o valor de $Y_1$.
		$\rho\in [-1,1]$ controla o grau de associação, ou correlação, entre $Y_1$ e $Y_2$.
		
		$\mu_1$ e $\mu_2$ são as esperanças MARGINAIS de $Y_1$ e $Y_2$.
		As esperanças podem ser condicionais, \textit{i.e.}, $\mathbb{E}(Y_1|Y_2=x)$ ou $\mathbb{E}(Y_2|Y_1=x)$.
		Essas esperanças provavelmente não serão iguais.
		A mesma análise vale para o desvio padrão $\sigma^2$.
		
		A distribuição da probabilidade condicional é uma normal: $(Y_2|Y_1=x)\approx N(\mu_{Y_2|Y_1=x},\sigma_{Y_2|Y_1=x}^2)$.
		Assim, obtemos uma fórmula geral para expressar qual é essa distribuição.
		Ela depende do coeficiente de correlação $\rho$:
		
		\begin{eqnarray*}
			(Y_2|Y_1=y)\sim N(\mu_{Y_2|Y_1=y},\sigma_{Y_2|Y_1=y}^2)\\
			com\\
			\mu_{Y_2|Y_1=y}=\mu_2+\frac{\rho\sigma_2}{\sigma_1}(y-\mu_1)\\
			e\\
			\sigma_{Y_2|Y_1=y}=\sigma_2\sqrt{1-\rho^2}
		\end{eqnarray*}
		
		\subsection{Matriz de covariância}
		
		Matriz 2x2 simétrica:
		
		\begin{eqnarray*}
			\sum = \left[
						\begin{tabular}{cc}
						$\sigma_x^2$ & $\rho\sigma_x\sigma_y$\\
						 $\rho\sigma_x\sigma_y$ & $\sigma_y^2$
						\end{tabular}
					\right]
		\end{eqnarray*}
		
		Onde $\rho\in [-1,1]$ e $\sigma_x$ e $\sigma_y$ são os desvios padrões de cada marginal.
		
		A fórmula geral de uma normal bivariada é igual a
		
		\begin{eqnarray*}
			f_\textbf{Y(y)}=cte\times e^{-\frac{1}{2}d^2(y,\mu)}
		\end{eqnarray*}
		
		onde $d^2(y,\mu)$ é a distância entre o ponto \textbf{y} e o vetor esperado $\mu$.
		Essa distância não é a euclidiana:
		
		$d^2(y,\mu)=(y-\mu)'\sum^{-1}(y-\mu)$
		sendo que:
		
		\begin{eqnarray*}
			\mu=(\mu_1,\mu_2)'=(\mathbb{E}(Y_1),\mathbb{E}(Y_2))
		\end{eqnarray*}
		
		é um vetor-coluna 2x1 das esperanças marginais.
		
		Um vetor normal multivariado tem uma densidade conjunta que é proporcional à explonencial de menos uma medida de distância ao quadrado.
		
		\begin{eqnarray*}
			f_\textbf{Y(y)}=cte\times e^{-\frac{1}{2}d^2(y,\mu)}
		\end{eqnarray*}
		
		A densidade decai exponencialmente à medida que a distância ao quadrado entre \textbf{y} e \textbf{$\mu$} aumenta.
		
		\subsection{Desvio padronizado}
		
		O desvio padronizado é definido como segue:
		
		\begin{eqnarray*}
			Z=\frac{Y-\mu}{\sigma}
		\end{eqnarray*}
		
		Ou seja, ele é medido relativamente ao desvio-padrão $\mu$ da v.a. Y.
		Se Z=2, significa um afastamento de 2 DPs em relação a $\mu$.
		
		Para medir a associação entre duas v.a.s $Y_1$ e $Y_2$, medidas num mesmo item, comparamos os desvios padronizados das duas, ou seja, comparamos $Z_1=\frac{Y_1-\mu_1}{\sigma_1}$ com $Z_2=\frac{Y_2-\mu_2}{\sigma_2}$.
		Se um for grande implicar numa tendencia do outro aumentar, então elas possuem correlação.
		
		Uma forma de medir a correlação é o índice de correlação de Pearson:
		
		\begin{eqnarray*}
			Z_1Z_2=\frac{Y_1-\mu_1}{\sigma_1}\times\frac{Y_2-\mu_2}{\sigma_2}
		\end{eqnarray*}
		
		Se desvios grandes e positivos (negativos) de $Y_1$ tendem a ocorrer com desvios grandes e positivos (negativos) de $Y_2$, seu produto será maior (menor) ainda.
		
		O produto $Z_1Z_2$ é uma v.a., logo:
		
		\begin{eqnarray*}
			\rho=Corr(Y_1,Y_2)=\mathbb{E}(Z_1Z_2)=\mathbb{E}(\frac{Y_1-\mu_1}{\sigma_1}\times\frac{Y_2-\mu_2}{\sigma_2})
		\end{eqnarray*}
		
		Pela definição: 
		
		\begin{eqnarray*}
			Corr(Y_1,Y_2)=&\mathbb{E}(Z_1Z_2)=Corr(Y_2,Y_1)\\
			Corr(Y,Y)=&1
		\end{eqnarray*}
		
		Se $Y_1$ é uma v.a. independente da v.a. $Y_2$, então $\rho=0$, formando um gráfico de dispersão como uma nuvem sem inclinação.
		
		Se $\rho\approx\pm 1$, então $Y_2$ é aproximadamente uma função linear perfeita de $Y_1$, \textit{i.e.}, os valores de ($Y_1,Y_2$) formarão  um gráfico de dispersão na forma aproximada de uma linha reta.
		
		\subsection{Matriz de correlação}
		
		Quando existem p v.a.s em um vetor, cria-se uma matriz de correlação pxp.
		Na posição (i,j) temos:
		
		\begin{eqnarray*}
			\rho_{ij}=Corr(Y_i,Y_j)=\mathbb{E}(\frac{Y_1-\mu_i}{\sigma_i}\times\frac{Y_j-\mu_j}{\sigma_j})
		\end{eqnarray*}
		
		Como $Corr(Y_i,Y_J)=Corr(Y_J,Y_i)$, a matriz é simétrica, e como $Corr(Y_i,Y_i)=1$, a diagonal principal é toda de 1's.
		
		$\rho$ é invariante por mudança linear de escala.
		Ou seja, se $Y_1,Y_2$ e $Y_3$ são v.a.s e $Y_3=2.3Y_2$, $Corr(Y_1,Y_2)=Corr(Y_1,2.3Y_3)=Corr(Y_1,Y_2)$
		
		Para estimar $\rho$, podemos utilizar as aproximações $\bar{Y}\approx\mathbb{E}(Y)$ e $S=\sqrt{\sum_i\frac{(Y_i-\bar{Y})^2}{n}}\approx\sigma$, ou seja:
		
		\begin{eqnarray*}
			\rho=\mathbb{E}(\frac{Y_1-\mu_1}{\sigma_1}\times\frac{Y_2-\mu_2}{\sigma_2})\approx\mathbb{E}(\frac{Y_1-\bar{Y_1}}{S_1}\times\frac{Y_2-\bar{Y_2}}{S_2})
		\end{eqnarray*}
		
		Para o desvio padronizado empírico, deve-se calcuar o desvio realizado de cada um dos \textit{n} valores das duas variáveis.
		Para a variável 1, com os \textit{n} valores $y_{11},\ldots,y_{n1}$ da coluna 1 da tabela, calcule a nova coluna formada por
		
		\begin{eqnarray*}
			z_{i1}=\frac{y_{i1}-\bar{y_1}}{S_1}
		\end{eqnarray*}
		
		Da mesma forma, pode-se calcular $z_{i2}$ para a coluna 2
		
		\begin{eqnarray*}
			z_{i2}=\frac{y_{i2}-\bar{y_2}}{S_2}
		\end{eqnarray*}
		
		Então, multiplique as duas colunas de desvios padronizados e tire a sua média aritmética:
		
		\begin{eqnarray*}
			r=\frac{1}{n}\sum_{i=1}^{n}z_{i1}z_{i2}=\frac{1}{n}\sum_{i=1}^{n}(\frac{y_{i1}-\bar{y_1}}{S_1})(\frac{y_{i2}-\bar{y_2}}{S_2})
		\end{eqnarray*}
		
		A distância estatística de pontos, indica o quanto o ponto se distanciou da distribuição, ou seja, não é a distância euclidiana.
		Esta distância pode ser vista como uma elipse em volta dos pontos.
		
		Para uma distância c>0 do centro, os pontos satisfazem a equação
		
		\begin{eqnarray*}
			d(y,\mu)=\sqrt{(\frac{y_1-\mu_1}{\sigma_1})^2+(\frac{y_2-\mu_2}{\sigma_2})^2}=c
		\end{eqnarray*}
		
		Os eixos da elipse possuem comprimentos iguais a $c\sigma_1$ e $c\sigma_2$.
		O maior eixo da elipse e a variável com maior DP.
		
		Quantas vezes maior é o eixo maior em relação ao menor não depende de c:
		
		\begin{eqnarray*}
			\frac{eixo maior}{eixo menor}=\frac{c\sigma_1}{c\sigma_2}=\frac{\sigma_1}{\sigma_2}
		\end{eqnarray*}
		
		Considerando que $\sigma_1$ é o maior.
		Variando c temos elipses concêntricas.
		
		\begin{eqnarray*}
			d^2(y,\mu) = & (\frac{y_1-\mu_1}{\sigma_1})^2+(\frac{y_2-\mu_2}{\sigma_2})^2\\
			=&(y_1-\mu_1,y_2-\mu_2)\left[
			\begin{tabular}{cc}
					$\frac{1}{\sigma_1^2}$ & 0\\
					0&$\frac{1}{\sigma_2^2}$
			\end{tabular}
			\right]
			\left(\begin{tabular}{c}
				$y_1-\mu_1$\\
				$y_2-\mu_2$
			\end{tabular}
			\right)\\
			=&(y_1-\mu_1,y_2-\mu_2)\left[
			\begin{tabular}{cc}
					$\sigma_1^2$ & 0\\
					0&$\sigma_2^2$
			\end{tabular}
			\right]^{-1}
			\left(\begin{tabular}{c}
				$y_1-\mu_1$\\
				$y_2-\mu_2$
			\end{tabular}
			\right)\\
			=&\left(\begin{tabular}{c}
				$y_1-\mu_1$\\
				$y_2-\mu_2$
			\end{tabular}
			\right)^{'}
			\sum^{-1}
			\left(\begin{tabular}{c}
				$y_1-\mu_1$\\
				$y_2-\mu_2$
			\end{tabular}
			\right)\\
			=&(y-\mu)^{'}\sum^{-1}(y-\mu)
		\end{eqnarray*}
		
		Neste caso, a matriz $\sum$ é diagonal com elementos positivos, então a elipse tem eixos paralelos aos eixos e o tamanho de cada eixo é proporcional ao $\sigma_1$ da variável associada.
		
		Num caso mais realista, a elipse está inclinada em relação aos eixos.
		Neste caso
		
		\begin{eqnarray*}
			\sum=\left[\begin{tabular}{cc}
			$\sigma_1^2$&$\rho\sigma_1\sigma_2$\\
			$\rho\sigma_1\sigma_2$&$\sigma_2^2$
			\end{tabular}
			\right]
		\end{eqnarray*}
		
		onde $\rho=Corr(Y_1,Y_2)$ é o índice de correlação de Pearson entre $Y_1$ e $Y_2$.
		Além disso, $-1\leq\rho\leq1$.
		Os elementos $\rho\sigma_1\sigma_2$ são chamados de Covariância entre $Y_1$ e $Y_2$.
		
		Costumamos escrever $Cov(Y_1,Y_2)=\rho\sigma_1\sigma_2=\sigma_{12}$.
		
		Eixos da elipse: na direção dos AUTOVETORES da matriz $\sum^{-1}$, e o tamanho de cada eixo é proporcional à raiz do AUTOVALOR correspondente.
		
		Autovetor de uma matriz quadrada A é um vetor \textbf{y} tal que
		
		\begin{eqnarray*}
			A\cdot\textbf{y}=\lambda\textbf{y}
		\end{eqnarray*}
		
		onde $\lambda$ é uma constante (autovalor associado ao autovetor \textbf{y}).
		
		Aqui, utilizando a matriz de covariância como $\sigma^{-1}$, temos dois autovetores ortogonais entre si e os autovalores são sempre reais e positivos.
\end{document}
