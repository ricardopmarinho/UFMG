\documentclass[11pt,a4paper]{book}
\usepackage[brazilian]{babel}
\usepackage[utf8]{inputenc}
\usepackage[T1]{fontenc}
\usepackage[inline]{enumitem}
\usepackage{xcolor}
\usepackage{listings}
\usepackage{graphicx}
\usepackage{multicol}
\usepackage{amsmath}

\definecolor{mGreen}{rgb}{0,0.6,0}
\definecolor{mGray}{rgb}{0.5,0.5,0.5}
\definecolor{mPurple}{rgb}{0.58,0,0.82}
\definecolor{backgroundColour}{rgb}{0.95,0.95,0.92}

\lstdefinestyle{CStyle}{
    backgroundcolor=\color{backgroundColour},   
    commentstyle=\color{mGreen},
    keywordstyle=\textbf{\color{black}},
    numberstyle=\tiny\color{mGray},
    stringstyle=\color{mPurple},
    basicstyle=\footnotesize,
    breakatwhitespace=false,         
    breaklines=true,                 
    captionpos=b,                    
    keepspaces=true,                 
    numbers=left,                    
    numbersep=5pt,                  
    showspaces=false,                
    showstringspaces=false,
    showtabs=false,                  
    tabsize=2,
    frame=single,
    escapeinside={(*}{*)},
    language=C
}

\makeatletter
% This command ignores the optional argument for itemize and enumerate lists
\newcommand{\inlineitem}[1][]{%
\ifnum\enit@type=\tw@
    {\descriptionlabel{#1}}
  \hspace{\labelsep}
\else
  \ifnum\enit@type=\z@
       \refstepcounter{\@listctr}\fi
    \quad\@itemlabel\hspace{\labelsep}
\fi}
\makeatother

\newcommand{\onestaritem}{\refstepcounter{enumi}\item[$*$\theenumi.]}
\newcommand{\twostaritem}{\refstepcounter{enumi}\item[$**$\theenumi.]}

\title{Lista 3: Fundamentos Estatísticos para Ciência dos Dados}
\author{Ricardo Pagoto Marinho}

\begin{document}
\maketitle
	\begin{enumerate}
		\item
			\begin{itemize}
				\item
					$P(A^{C})=1-P(A)$
			
					Seja $P(A)= q$ e $A^{C}$ os elementos que estão em $\Omega$ mas não em A.
					Logo, $A^{C}=\Omega-A$, então $P(A^{C})=P(\Omega)-P(A)=1-P(A)$.
			
				\item
					Sabe-se que $P(A)= 1$ e $\forall$ A $\subset A$, P($\Sigma \cup $A)=1.
					Logo, $0 \leq P(A) \leq 1$
				\item
					Suponha que $A_1 \subset A_2$ e $P(A_1) > P(A_2)$.
					Logo, dentro do experimento, existem mais possibilidades de aparecer um elemento de $A_1$ do que de $A_2$.
					Portanto, $A_1>A_2$.
					Absurdo, já que $A_1 \subset A_2$.
				\item
					Seja $A_i$ uma coleção de conjuntos de A.
					$P(\cup_{n=1}^{\infty} A_i)$ cai em dois casos:
					\begin{itemize}
						\item caso 1: Todos $A_is$ são disjuntos.
						Portanto, $P(\cup_{n=1}^{\infty} A_i)= \sum_{n=1}^{\infty} A_i $
						\item caso 2: Alguns $A_i$ se intersectam.
						Neste caso, deve-se tirar a probabilidade $P(A_i \cap A_{i+1}) $ para os $A_i$ e $A_{i+1}$ tais que $A_i \cap A_{i+1} \neq \emptyset$.
						Logo $P(\cup_{n=1}^{\infty} A_i) < \sum_{n=1}^{\infty} A_i $
				
						Portanto $P(\cup_{n=1}^{\infty} A_i)\leq \sum_{n=1}^{\infty} A_i $.
					\end{itemize}
				\item $P(A \cup B)$ é a probabilidade de um item selecionado estiver em $A$ ou $B$.
			C	ontudo, caso $A \cap B \neq \emptyset$, a soma das probabilidades somará duas vezes os itens que estão em $A \cap B$.
				Logo, $P(A \cup B) = P(A) + P(B) - P(A \cap B)$
			\end{itemize}
		\item
			\begin{itemize}
				\item V
				\item V
				\item F
				\item F
				\item F
				\item V
				\item V
				\item V
			\end{itemize}
		\item O contrário não pode ser dito.
		
		$P(B|A)=\frac{P(B \cap A)}{P(A)}$, caso $A \supset B$, $P(B \cap A) = P(B)$ e $P(B|A)=\frac{P(B)}{P(A)}$.
		Como $0 \leq P(A)\leq 1$, $\frac{P(B)}{P(A)} \geq P(B)$.
		
		\item Essas probabilidades podem ser obtidas observando os pacientes ao longo de um ano e contabiliza os pacientes que se mantiveram vivos e os que não sobreviveram ao período.
		
		\item 
			\begin{itemize}
				\item Significa que, quanto maior a taxa, mais confiável fica o teste, logo todos os positivos serão flagrados.
				\item A afirmação diz que quanto maior a sensibilidade, mais confiável o teste se torna.
				\item \textit{Recall} é proporção de pacientes que o teste foi positivo e ele realmente estava doente.
			\end{itemize}
		\item
			To do.
		\item
			\begin{itemize}
				\item F
				\item F
				\item F
				\item V
			\end{itemize}
		\item Para mostrar que um evento é independente de outro, temos que mostrar que $P(A|B)=P(A)\times P(B)$.
		Como $P(A)=0,~P(A|B)=0\times P(B)=0, \forall B$, logo A é independente de qualquer B quando $P(A)=0$.
		
		Sim, intuitivamente faz sentido, já que se não existe a possibilidade de A, os eventos seguintes não vão sofrer nenhuma influência dele.
		
		\item $P(A|A)=P(A)\times P(A) \longleftrightarrow P(A)=1~ou~P(A)=0$.
	\end{enumerate}
\end{document}